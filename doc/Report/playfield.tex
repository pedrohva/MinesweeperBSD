%----------------------------------------------------------------------------------------
%	REPRESENTING THE PLAYFIELD
%----------------------------------------------------------------------------------------
The playfield is represented as a two-dimensional array of \textit{Tile} objects. 
\begin{lstlisting}[style=CStyle]
// minesweeper.h: Line 26
Tile field[FIELD_WIDTH][FIELD_HEIGHT];
\end{lstlisting}
Each \textit{Tile} object is a struct that holds information that helps control the game.
\begin{lstlisting}[style=CStyle]
// minesweeper.h: Lines 15-20
typedef struct {
	int adjacent_mines;
	int revealed;
	int has_mine;
	int has_flag;
} Tile;
\end{lstlisting}
A two-dimensional array was chosen to save programming time. The ability to access the field directly through \textit{x-y} coordinates (\lstinline{field[x][y]}) greatly increases readability of the program. 
\\
There is some increased computational complexity to iterate through the field when compared with a one-dimensional array but since the field only contains 81 tiles, it's not enough of a drawback to sacrifice program readability. 